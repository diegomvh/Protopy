El desarrollo de un framework en javascript que funcione del lado del cliente presupone gran cantidad de codigo
viajando de un lado al otro de la conexion. Previendo basicamente este postulado desarrollamos una libreria
que brinde el soporte a los requerimentos mas basicos.
Esta librería contituye la base para posteriores contrucciones mas complejas en el ciente y auna herramientas
que simplifican el desarrollo client-side.

\section{Modulos}
Undo de los principales inconvenientes a los que protopy da solucion es a la inclucion dinamica de funcionalidad bajo demanda,
esto es logrado mediante los modulos.
Basicamente un modulo en un archivo con codigo javascript que recide en el servidor y es obtenido y ejecutado en el cliente.

\begin{lstlisting}[style=javascript,label=estructura-modulo,caption=Estructura de un modulo]
//Archivo: tests/module.js
require('event');

var h1 = $('titulo');

function set_texto(txt) {
    h1.update(txt);
}

function get_texto() {
    return h1.innerHTML;
}

event.connect($('titulo'), 'click', function(event) {
    alert('El texto es: ' + event.target.innerHTML);
});

publish({
    set_texto: set_texto,
    get_texto: get_texto
});
\end{lstlisting}

\begin{lstlisting}[style=consola,label=estructura-modulo-test,caption=Test]
>>> require('tests.module')
GET http://localhost:8080/tests/protopy/module.js
Object __file__=/tests/protopy/module.js
>>> module.get_texto()
"Test de modulo"
>>> module.set_texto('Un titulo')
>>> require('tests.module', 'get_texto')
get_texto()
>>> get_texto()
"Un titulo"
>>> require('tests.module', '*')
>>> set_texto('Hola luuu!!!')
>>> get_texto()
"Hola luuu!!!"
\end{lstlisting}

\subsection{builtin}
\subsection{sys}
\subsection{exception}
\subsection{event}
\subsection{timer}
\subsection{ajax}
\subsection{dom}

\section{sys}
\subsection*{version}
version: 0.8,
\subsection*{browser}
	browser: {
	    IE:     !!(window.attachEvent && navigator.userAgent.indexOf('Opera') === -1),
	    Opera:  navigator.userAgent.indexOf('Opera') > -1,
	    WebKit: navigator.userAgent.indexOf('AppleWebKit/') > -1,
	    Gecko:  navigator.userAgent.indexOf('Gecko') > -1 && navigator.userAgent.indexOf('KHTML') === -1,
	    MobileSafari: !!navigator.userAgent.match(/Apple.*Mobile.*Safari/),
	    features: {
		XPath: !!document.evaluate,
		SelectorsAPI: !!document.querySelector,
		ElementExtensions: !!window.HTMLElement,
		SpecificElementExtensions: document.createElement('div')['__proto__'] &&
						document.createElement('div')['__proto__'] !==
						document.createElement('form')['__proto__']
	      Gears = !!get_gears() || false;

	    }
	},
\subsection*{get\_gears}
get\_gears: get_gears,
\subsection*{register\_path}
register\_path: function(module, path)
\subsection*{module\_url}
module\_url: function(name, postfix) {
\subsection*{modules}
modules: ModuleManager.modules,
\subsection*{paths}
paths: ModuleManager.paths

\section{exception}
var exception = ModuleManager.create('exceptions', 'built-in', {
        Exception: Exception,
        AssertionError: type('AssertionError', Exception),
        AttributeError: type('AttributeError', Exception),
        LoadError: type('LoadError', Exception),
        KeyError: type('KeyError', Exception),
        NotImplementedError: type('NotImplementedError', Exception),
        TypeError: type('TypeError', Exception),
        ValueError: type('ValueError', Exception),
    });

\section{event}
\subsection*{connect}
connect(object, event, context, method) {
\subsection*{disconnect}
disconnect(handle) {
\subsection*{subscribe}
subscribe(topic, context, method) {
\subsection*{unsubscribe}
unsubscribe(handle) {
\subsection*{publish}
publish(topic, args) {
\subsection*{connectpublisher}
connectpublisher(topic, obj, event) {
\subsection*{fixevent}
fixevent: function(){},
\subsection*{stopevent}
stopevent: function(){},
\subsection*{keys}
keys: { BACKSPACE: 8, TAB: 9, CLEAR: 12, ENTER: 13, SHIFT: 16, CTRL: 17, ALT: 18, PAUSE: 19, CAPS_LOCK: 20, 
		    ESCAPE: 27, SPACE: 32, PAGE_UP: 33, PAGE_DOWN: 34, END: 35, HOME: 36, LEFT_ARROW: 37, UP_ARROW: 38,
		    RIGHT_ARROW: 39, DOWN_ARROW: 40, INSERT: 45, DELETE: 46, HELP: 47, LEFT_WINDOW: 91, RIGHT_WINDOW: 92,
		    SELECT: 93, NUMPAD_0: 96, NUMPAD_1: 97, NUMPAD_2: 98, NUMPAD_3: 99, NUMPAD_4: 100, NUMPAD_5: 101,
		    NUMPAD_6: 102, NUMPAD_7: 103, NUMPAD_8: 104, NUMPAD_9: 105, NUMPAD_MULTIPLY: 106, NUMPAD_PLUS: 107,
		    NUMPAD_ENTER: 108, NUMPAD_MINUS: 109, NUMPAD_PERIOD: 110, NUMPAD_DIVIDE: 111, F1: 112, F2: 113, F3: 114,
		    F4: 115, F5: 116, F6: 117, F7: 118, F8: 119, F9: 120, F10: 121, F11: 122, F12: 123, F13: 124, 
		    F14: 125, F15: 126, NUM_LOCK: 144, SCROLL_LOCK: 145 }
    });

\section{timer}
\subsection*{setTimeout}
setTimeout: window.setTimeout,
\subsection*{setInterval}
setInterval: window.setInterval,
\subsection*{clearTimeout}
clearTimeout: window.clearTimeout,
\subsection*{delay}
delay: function(f) {
\subsection*{defer}
defer: function(f) {

\section{ajax}
\subsection*{Request}
new Request: Request,
\subsection*{Response}
new Response: Response,
\subsection*{toQueryParams}
toQueryParams(string, separator) -> object
\subsection*{toQueryString}
toQueryString(params) -> string

\section{dom}
\subsection*{query}
query: query,

\section{builtin}
\subsection*{publish}
publish(object)
\subsection*{require}
require(module[, object...]) -> module | object
\subsection*{\$}
\$(id) -> HTMLElement
\$(id ...) -> [HTMLElement...]
\subsection*{\$\$}
\$\$(cssRule) -> [HTMLElement...]
\subsection*{type, El constructor de tipos}
type(name, [base...] [, forType ], forPrototype) -> newType

Como ya se menciono anteriormente javascript es un lenguaje orientado a prototipos, para acercarnos un poco a la programacion
de objetos, utilizamos una funcion constructora de tipos o clases a la que denominamos “type”.
\begin{lstlisting}[style=javascript,label=definicion-de-tipos,caption=Definicion de tipos]
var Animal = type('Animal', object, {
    contador: 0,
}, {
    __init__: function(especie) {
	this.especie = especie;
	this.orden = Animal.contador++;
    }
});

var Terrestre = type('Terrestre', Animal, {
    caminar: function() {
	console.log(this.especie + ' caminando');
    }
});

var Acuatico = type('Acuatico', Animal, {
    nadar: function() {
	console.log(this.especie + ' nadando');
    }
});

var Anfibio = type('Anfibio', [Terrestre, Acuatico]);

var Piton = type('Piton', Terrestre, {
    __init__: function(nombre) {
	super(Terrestre, this).__init__(this.__name__);
	this.nombre = nombre;
    },
    caminar: function() {
	throw new Exception(this.especie + ' no camina');
    },
    reptar: function() {
	console.log(this.nombre + ' la ' + this.especie.toLowerCase() + ' esta reptando');
    }
});

var doris = new Piton('Doris');
var ballena = new Acuatico('Ballena');
var rana = new Anfibio('Rana');
\end{lstlisting}

\begin{lstlisting}[style=consola,label=definicion-de-tipos-test,caption=Test]
>>> doris
window.Piton especie=Piton orden=0 nombre=Doris __name__=Piton
>>> rana
window.Anfibio especie=Rana orden=2 __name__=Anfibio
>>> isinstance(rana, Terrestre)
true
>>> isinstance(doris, Animal)
true
>>> issubclass(Anfibio, Acuatico)
true
>>> issubclass(Piton, Animal)
true
>>> doris.caminar()
Exception: Piton no camina args=[1] message=Piton no camina
>>> doris.reptar()
Doris la piton esta reptando
\end{lstlisting}

\subsection*{extend}
extend(destiny, source) -> alteredDestiny
\subsection*{super}
super(destiny, source) -> alteredDestiny
\subsection*{isundefined}
isundefined(destiny, source) -> alteredDestiny
\subsection*{isinstance}
isundefined(destiny, source) -> alteredDestiny
\subsection*{issubclass}
issubclass(destiny, source) -> alteredDestiny
\subsection*{Arguments}
new Arguments(destiny, source) -> alteredDestiny
\subsection*{Template}
new Template(destiny, source) -> alteredDestiny
\subsection*{Dict}
new Dict(destiny, source) -> alteredDestiny
\subsection*{Set}
new Set(destiny, source) -> alteredDestiny
\subsection*{hash}
hash(destiny, source) -> alteredDestiny
\subsection*{id}
id(destiny, source) -> alteredDestiny
\subsection*{getattr}
getattr(destiny, source) -> alteredDestiny
\subsection*{setattr}
setattr(destiny, source) -> alteredDestiny
\subsection*{hasattr}
hasattr(destiny, source) -> alteredDestiny
\subsection*{assert}
assert(destiny, source) -> alteredDestiny
\subsection*{bool}
bool(destiny, source) -> alteredDestiny
\subsection*{callable}
callable(destiny, source) -> alteredDestiny
\subsection*{chr}
chr(destiny, source) -> alteredDestiny
\subsection*{ord}
ord(destiny, source) -> alteredDestiny
\subsection*{bisect}
bisect(destiny, source) -> alteredDestiny
\subsection*{equal}
equal(destiny, source) -> alteredDestiny
\subsection*{nequal}
nequal(destiny, source) -> alteredDestiny
\subsection*{number}
number(destiny, source) -> alteredDestiny
\subsection*{flatten}
flatten(destiny, source) -> alteredDestiny
\subsection*{include}
include(destiny, source) -> alteredDestiny
\subsection*{len}
len(destiny, source) -> alteredDestiny
\subsection*{array}
array(iterable) -> [element...]
\subsection*{print}
print(destiny, source) -> alteredDestiny
\subsection*{range}
range(destiny, source) -> alteredDestiny
\subsection*{string}
string(destiny, source) -> alteredDestiny
\subsection*{values}
values(object) -> [value...]
\subsection*{keys}
keys(object) -> [key...]
\subsection*{items}
items(destiny, source) -> alteredDestiny
\subsection*{inspect}
inspect(destiny, source) -> alteredDestiny
\subsection*{unique}
unique(destiny, source) -> alteredDestiny
\subsection*{xrange}
xrange(destiny, source) -> alteredDestiny
\subsection*{zip}
zip(destiny, source) -> alteredDestiny

santiago del estero norte 727, tiene un pino grande

\section{Extendiendo Javascript}
\subsection{String}
\subsubsection*{gsub}
gsub(pattern, replacement) -> string
\subsubsection*{sub}
sub(pattern, replacement[, count = 1]) -> string
\subsubsection*{subs}
subs(pattern, replacement) -> string
\subsubsection*{format}
format(pattern, replacement) -> string
\subsubsection*{inspect} 
inspect(use_double_quotes) -> string
\subsubsection*{truncate}
truncate(length, truncation) -> string
\subsubsection*{strip}
strip() -> string
\subsubsection*{striptags}
striptags() -> string
\subsubsection*{stripscripts}
stripscripts() -> string
\subsubsection*{extractscripts}
extractscripts() -> string
\subsubsection*{evalscripts}
evalscripts() -> string
\subsubsection*{escapeHTML}
escapeHTML() -> string
\subsubsection*{unescapeHTML}
unescapeHTML() -> string
\subsubsection*{succ}
succ() -> string
\subsubsection*{times}
times(count[, separator = '']) -> string
\subsubsection*{camelize}
camelize() -> string
\subsubsection*{capitalize}
capitalize() -> string
\subsubsection*{underscore}
underscore() -> string
\subsubsection*{dasherize}
dasherize() -> string
\subsubsection*{startswith}
startswith(pattern) -> string
\subsubsection*{endswith}
endswith(pattern) -> string
\subsubsection*{blank}
blank() -> string

\subsection{Number}
\subsubsection*{format}
format(f, radix) -> string

\subsection{Date}
\subsubsection*{toISO8601}
toISO8601() -> string

\subsection{Element}
\subsubsection*{visible}
visible() {
\subsubsection*{toggle}
toggle() {
\subsubsection*{hide}
hide() {
\subsubsection*{show}
show() {
\subsubsection*{remove}
remove() {
\subsubsection*{update}
update(content) {
\subsubsection*{insert}
insert(insertions) {
\subsubsection*{select}
select(selector) {

\subsection{Forms}
\subsubsection*{disable}
disable() {
\subsubsection*{enable}
enable() {
\subsubsection*{serialize}
serialize() {

\subsection{Forms.Element}
\subsubsection*{serialize}
serialize() {
\subsubsection*{get\_value}
get\_value() {
\subsubsection*{set\_value}
set\_value(value) {
\subsubsection*{clear}
clear() {
\subsubsection*{present}
present() {
\subsubsection*{activate}
activate() {
\subsubsection*{disable}
disable() {
\subsubsection*{enable}
enable() {