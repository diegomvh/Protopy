El desarrollo de un framework en javascript que funcione del lado del cliente presupone gran cantidad de codigo
viajando de un lado al otro de la conexion. Previendo basicamente este postulado desarrollamos una libreria
que brinde el soporte a los requerimentos mas basicos.
Esta librería contituye la base para posteriores contrucciones mas complejas en el ciente y auna herramientas
que simplifican el desarrollo client-side.

\section{Modulos}
Undo de los principales inconvenientes a los que protopy da solucion es a la inclucion dinamica de funcionalidad bajo demanda,
esto es logrado mediante los modulos.
Basicamente un modulo en un archivo con codigo javascript que recide en el servidor y es obtenido y ejecutado en el cliente.

\begin{lstlisting}[style=javascript,label=estructura-modulo,caption=Estructura de un modulo]
//Archivo: tests/module.js
require('event');

var h1 = $('titulo');

function set_texto(txt) {
    h1.update(txt);
}

function get_texto() {
    return h1.innerHTML;
}

event.connect($('titulo'), 'click', function(event) {
    alert('El texto es: ' + event.target.innerHTML);
});

publish({
    set_texto: set_texto,
    get_texto: get_texto
});
\end{lstlisting}

\noindent
Probamos el modulo:
\begin{lstlisting}[style=consola]
>>> require('tests.module')
>>> module.get_texto()
"Test de modulo"
>>> module.set_texto('Un titulo')
>>> require('tests.module', 'get_texto')
>>> get_texto()
"Un titulo"
>>> require('tests.module', '*')
>>> set_texto('Hola luuu!!!')
>>> get_texto()
"Hola luuu!!!"
\end{lstlisting}

\section{Modulos incluidos en el nucleo de protopy}
\subsection{builtin}
\subsubsection*{publish}
\verb|publish(object)|

\subsubsection*{require}
\verb|require(module[, object...]) -> module or object|

\subsubsection*{type}
\verb|type(name, [base...] [, forType ], forPrototype) -> newType|

Como ya se menciono anteriormente javascript es un lenguaje orientado a prototipos, para acercarnos un poco a la programacion
de objetos, utilizamos una funcion constructora de tipos o clases a la que denominamos “type”.
\begin{lstlisting}[style=javascript,label=definicion-de-tipos,caption=Definicion de tipos]
var Animal = type('Animal', object, {
    contador: 0,
}, {
    __init__: function(especie) {
	this.especie = especie;
	this.orden = Animal.contador++;
    }
});

var Terrestre = type('Terrestre', Animal, {
    caminar: function() {
	console.log(this.especie + ' caminando');
    }
});

var Acuatico = type('Acuatico', Animal, {
    nadar: function() {
	console.log(this.especie + ' nadando');
    }
});

var Anfibio = type('Anfibio', [Terrestre, Acuatico]);

var Piton = type('Piton', Terrestre, {
    __init__: function(nombre) {
	super(Terrestre, this).__init__(this.__name__);
	this.nombre = nombre;
    },
    caminar: function() {
	throw new Exception(this.especie + ' no camina');
    },
    reptar: function() {
	console.log(this.nombre + ' la ' + this.especie.toLowerCase() + ' esta reptando');
    }
});

var doris = new Piton('Doris');
var ballena = new Acuatico('Ballena');
var rana = new Anfibio('Rana');
\end{lstlisting}

\noindent
Ejemplos con los tipos definidos:
\begin{lstlisting}[style=consola]
>>> doris
window.Piton especie=Piton orden=0 nombre=Doris __name__=Piton
>>> rana
window.Anfibio especie=Rana orden=2 __name__=Anfibio
>>> isinstance(rana, Terrestre)
true
>>> isinstance(doris, Animal)
true
>>> issubclass(Anfibio, Acuatico)
true
>>> issubclass(Piton, Animal)
true
>>> doris.caminar()
Exception: Piton no camina args=[1] message=Piton no camina
>>> doris.reptar()
Doris la piton esta reptando
\end{lstlisting}

\subsubsection*{\$}
\$(id) \rightarrow HTMLElement
\$(id ...) \rightarrow [HTMLElement...]

\begin{lstlisting}[style=consola]
>>> $('content')
<div id="content">
>>> $('content body')
>>> $('content', 'body')
[div#content, div#body]
>>> $('content', 'body', 'head')
[div#content, div#body, undefined]
\end{lstlisting}

\subsubsection*{\$\$}
\$\$(cssRule) \rightarrow [HTMLElement...]

\noindent
Ejemplo:
\begin{lstlisting}[style=consola]
>>> $$('div')
[div#wrap, div#top, div#content, div.header, div.breadcrumbs, div.middle, div, div.right, div#clear, div#footer, div#toolbar]
>>> $$('div#toolbar')
[div#toolbar]
>>> $$('div#toolbar li')
[li, li.panel, li.panel, li, li]
>>> $$('div#toolbar li.panel')
[li.panel, li.panel]
>>> $$('a:not([href~=google])')
[a add_post, a add_tag, a removedb, a syncdb]
>>> $$('a:not([href=google])')
[a add_post, a add_tag, a#google www.google.com, a removedb, a syncdb]
>>> $$('div:empty')
[div#logger.panel, div#dbquery.panel, div#clear, div#top]
\end{lstlisting}

\subsubsection*{extend}
\verb|extend(destiny, source) -> alteredDestiny|

\begin{lstlisting}[style=consola]
>>> a = {perro: 4}
>>> b = {gato: 4}
>>> c = extend(a, b)
>>> c
Object perro=4 gato=4
>>> a
Object perro=4 gato=4
>>> b
Object gato=4
\end{lstlisting}

\subsubsection*{super}
super(destiny, source) \rightarrow alteredDestiny
\subsubsection*{isundefined}
isundefined(destiny, source) \rightarrow alteredDestiny
\subsubsection*{isinstance}
isundefined(destiny, source) \rightarrow alteredDestiny
\subsubsection*{issubclass}
issubclass(destiny, source) \rightarrow alteredDestiny
\subsubsection*{Arguments}
new Arguments(destiny, source) \rightarrow alteredDestiny

\begin{lstlisting}[style=consola]
\end{lstlisting}

\subsubsection*{Template}
new Template(destiny, source) \rightarrow alteredDestiny
\subsubsection*{Dict}
new Dict(destiny, source) \rightarrow alteredDestiny
\subsubsection*{Set}
new Set(destiny, source) \rightarrow alteredDestiny
\subsubsection*{hash}
hash(destiny, source) \rightarrow alteredDestiny
\subsubsection*{id}
id(destiny, source) \rightarrow alteredDestiny
\subsubsection*{getattr}
getattr(destiny, source) \rightarrow alteredDestiny
\subsubsection*{setattr}
setattr(destiny, source) \rightarrow alteredDestiny
\subsubsection*{hasattr}
hasattr(destiny, source) \rightarrow alteredDestiny
\subsubsection*{assert}
assert(destiny, source) \rightarrow alteredDestiny
\subsubsection*{bool}
bool(destiny, source) \rightarrow alteredDestiny
\subsubsection*{callable}
callable(destiny, source) \rightarrow alteredDestiny
\subsubsection*{chr}
chr(number) \rightarrow character
\subsubsection*{ord}
ord(character) \rightarrow number

\noindent
Ejemplo:
\begin{lstlisting}[style=consola]
>>> ord(chr(65))
65
>>> chr(ord("A"))
"A"
\end{lstlisting}

\subsubsection*{bisect}
bisect(seq, element) \rightarrow position

\begin{lstlisting}[style=consola]
>>> a = [1,2,3,4,5]
>>> bisect(a,6)
5
>>> bisect(a,2)
2
>>> a[bisect(a,3)] = 3
>>> a
[1, 2, 3, 3, 5]
\end{lstlisting}

\subsubsection*{equal}
equal(destiny, source) \rightarrow alteredDestiny
\subsubsection*{nequal}
nequal(destiny, source) \rightarrow alteredDestiny
\subsubsection*{number}
number(destiny, source) \rightarrow alteredDestiny
\subsubsection*{flatten}
flatten(destiny, source) \rightarrow alteredDestiny
\subsubsection*{include}
include(destiny, source) \rightarrow alteredDestiny
\subsubsection*{len}
len(destiny, source) \rightarrow alteredDestiny
\subsubsection*{array}
array(iterable) \rightarrow [element...]
\subsubsection*{print}
print(destiny, source) \rightarrow alteredDestiny
\subsubsection*{string}
string(destiny, source) \rightarrow alteredDestiny
\subsubsection*{values}
values(object) \rightarrow [value...]
\subsubsection*{keys}
keys(object) \rightarrow [key...]
\subsubsection*{items}
items(object) \rightarrow [[key, value]...]

\noindent
Ejemplo:
\begin{lstlisting}[style=consola]
>>> items({'perro': 1, 'gato': 7})
[["perro", 1], ["gato", 7]]
\end{lstlisting}
\subsubsection*{inspect}
inspect(destiny, source) \rightarrow alteredDestiny
\subsubsection*{unique}
unique(destiny, source) \rightarrow alteredDestiny
\subsubsection*{range}
range([begin = 1, ] end[, step = 1]) \rightarrow [number...]

\begin{lstlisting}[style=consola]
>>> range(10)
[0, 1, 2, 3, 4, 5, 6, 7, 8, 9]
>>> range(4, 10)
[4, 5, 6, 7, 8, 9]
>>> range(4, 10, 2)
[4, 6, 8]
\end{lstlisting}

\subsubsection*{xrange}
xrange(destiny, source) \rightarrow alteredDestiny
\subsubsection*{zip}
zip(seq1 [, seq2 [...]]) \rightarrow [[seq1[0], seq2[0] ...], [...]]

\begin{lstlisting}[style=consola]
>>> zip([1,2,3,4,5,6], ['a','b','c','d','e','f'])
[[1, "a"], [2, "b"], [3, "c"], [4, "d"], [5, "e"], [6, "f"]]
>>> zip([1,2,3,4,5,6], ['a','b','c','d','e','f','g','h'])
[[1, "a"], [2, "b"], [3, "c"], [4, "d"], [5, "e"], [6, "f"]]
>>> zip([1,2,3,4,5,6], ['a','b','c','d'])
[[1, "a"], [2, "b"], [3, "c"], [4, "d"], [5, undefined], [6, undefined]]
>>> zip([1,2,3,4,5,6], ['a','b','c','d','e','f'], [10,11,12,13,14,15,16])
[[1, "a", 10], [2, "b", 11], [3, "c", 12], [4, "d", 13], [5, "e", 14], [6, "f", 15]]
\end{lstlisting}

\subsection{sys}
\subsubsection*{version}
version: 0.8,
\subsubsection*{browser}
	browser: {
	    IE:     !!(window.attachEvent && navigator.userAgent.indexOf('Opera') === -1),
	    Opera:  navigator.userAgent.indexOf('Opera') > -1,
	    WebKit: navigator.userAgent.indexOf('AppleWebKit/') > -1,
	    Gecko:  navigator.userAgent.indexOf('Gecko') > -1 && navigator.userAgent.indexOf('KHTML') === -1,
	    MobileSafari: !!navigator.userAgent.match(/Apple.*Mobile.*Safari/),
	    features: {
		XPath: !!document.evaluate,
		SelectorsAPI: !!document.querySelector,
		ElementExtensions: !!window.HTMLElement,
		SpecificElementExtensions: document.createElement('div')['__proto__'] &&
						document.createElement('div')['__proto__'] !==
						document.createElement('form')['__proto__']
	      Gears = !!get_gears() || false;

	    }
	},
\subsubsection*{get\_gears}
get\_gears: get_gears,
\subsubsection*{register\_path}
register\_path: function(module, path)
\subsubsection*{module\_url}
module\_url: function(name, postfix) {
\subsubsection*{modules}
modules: ModuleManager.modules,
\subsubsection*{paths}
paths: ModuleManager.paths

\subsection{exception}
var exception = ModuleManager.create('exceptions', 'built-in', {
        Exception: Exception,
        AssertionError: type('AssertionError', Exception),
        AttributeError: type('AttributeError', Exception),
        LoadError: type('LoadError', Exception),
        KeyError: type('KeyError', Exception),
        NotImplementedError: type('NotImplementedError', Exception),
        TypeError: type('TypeError', Exception),
        ValueError: type('ValueError', Exception),
    });

\subsection{event}
\subsubsection*{connect}
connect(object, event, context, method)
\subsubsection*{disconnect}
disconnect(handle)
\subsubsection*{subscribe}
subscribe(topic, context, method)
\subsubsection*{unsubscribe}
unsubscribe(handle)
\subsubsection*{publish}
publish(topic, args)
\subsubsection*{connectpublisher}
connectpublisher(topic, obj, event)
\subsubsection*{fixevent}
fixevent()
\subsubsection*{stopevent}
stopevent()
\subsubsection*{keys}
keys: { BACKSPACE: 8, TAB: 9, CLEAR: 12, ENTER: 13, SHIFT: 16, CTRL: 17, ALT: 18, PAUSE: 19, CAPS_LOCK: 20, 
		    ESCAPE: 27, SPACE: 32, PAGE_UP: 33, PAGE_DOWN: 34, END: 35, HOME: 36, LEFT_ARROW: 37, UP_ARROW: 38,
		    RIGHT_ARROW: 39, DOWN_ARROW: 40, INSERT: 45, DELETE: 46, HELP: 47, LEFT_WINDOW: 91, RIGHT_WINDOW: 92,
		    SELECT: 93, NUMPAD_0: 96, NUMPAD_1: 97, NUMPAD_2: 98, NUMPAD_3: 99, NUMPAD_4: 100, NUMPAD_5: 101,
		    NUMPAD_6: 102, NUMPAD_7: 103, NUMPAD_8: 104, NUMPAD_9: 105, NUMPAD_MULTIPLY: 106, NUMPAD_PLUS: 107,
		    NUMPAD_ENTER: 108, NUMPAD_MINUS: 109, NUMPAD_PERIOD: 110, NUMPAD_DIVIDE: 111, F1: 112, F2: 113, F3: 114,
		    F4: 115, F5: 116, F6: 117, F7: 118, F8: 119, F9: 120, F10: 121, F11: 122, F12: 123, F13: 124, 
		    F14: 125, F15: 126, NUM_LOCK: 144, SCROLL_LOCK: 145 }
    });

\subsection{timer}
\subsubsection*{setTimeout}
setTimeout: window.setTimeout,
\subsubsection*{setInterval}
setInterval: window.setInterval,
\subsubsection*{clearTimeout}
clearTimeout: window.clearTimeout,
\subsubsection*{delay}
delay: function(f) {
\subsubsection*{defer}
defer: function(f) {

\subsection{ajax}
\subsubsection*{Request}
new Request()
\subsubsection*{Response}
new Response()
\subsubsection*{toQueryParams}
toQueryParams(string, separator) \rightarrow object
\subsubsection*{toQueryString}
toQueryString(params) \rightarrow string

\subsection{dom}
\subsubsection*{query}
query(cssRule) \rightarrow [HTMLElement...]

\section{Extendiendo Javascript}
\subsection{String}
\subsubsection*{gsub}
gsub(pattern, replacement) \rightarrow string
\subsubsection*{sub}
sub(pattern, replacement[, count = 1]) \rightarrow string
\subsubsection*{subs}
subs(pattern, replacement) \rightarrow string
\subsubsection*{format}
format(pattern, replacement) \rightarrow string
\subsubsection*{inspect} 
inspect(use_double_quotes) \rightarrow string
\subsubsection*{truncate}
truncate(length, truncation) \rightarrow string
\subsubsection*{strip}
strip() \rightarrow string
\subsubsection*{striptags}
striptags() \rightarrow string
\subsubsection*{stripscripts}
stripscripts() \rightarrow string
\subsubsection*{extractscripts}
extractscripts() \rightarrow string
\subsubsection*{evalscripts}
evalscripts() \rightarrow string
\subsubsection*{escapeHTML}
escapeHTML() \rightarrow string
\subsubsection*{unescapeHTML}
unescapeHTML() \rightarrow string
\subsubsection*{succ}
succ() \rightarrow string
\subsubsection*{times}
times(count[, separator = '']) \rightarrow string
\subsubsection*{camelize}
camelize() \rightarrow string
\subsubsection*{capitalize}
capitalize() \rightarrow string
\subsubsection*{underscore}
underscore() \rightarrow string
\subsubsection*{dasherize}
dasherize() \rightarrow string
\subsubsection*{startswith}
startswith(pattern) \rightarrow string
\subsubsection*{endswith}
endswith(pattern) \rightarrow string
\subsubsection*{blank}
blank() \rightarrow string

\subsection{Number}
\subsubsection*{format}
format(f, radix) \rightarrow string

\subsection{Date}
\subsubsection*{toISO8601}
toISO8601() \rightarrow string

\subsection{Element}
\subsubsection*{visible}
visible()
\subsubsection*{toggle}
toggle()
\subsubsection*{hide}
hide()
\subsubsection*{show}
show()
\subsubsection*{remove}
remove()
\subsubsection*{update}
update(content)
\subsubsection*{insert}
insert(insertions)
\subsubsection*{select}
select(selector)

\begin{lstlisting}[style=consola]
>>> $('PostForm').select('input')
[input#id_title, input guardar]
>>> $('content').select('div')
[div.header, div.breadcrumbs, div.middle, div, div.right, div#clear]
>>> $('content').select('div.middle')
[div.middle]
\end{lstlisting}

\subsection{Forms}
\subsubsection*{disable}
disable()
\subsubsection*{enable}
enable()
\subsubsection*{serialize}
serialize() \rightarrow object
\begin{lstlisting}[style=consola]
>>> $('PostForm')
<form id="PostForm" method="post" onsubmit="return blog.handler.handle(this);" action="/blog/add_post/">
>>> $('PostForm').serialize()
Object title=Hola mundo body=Este es un post tags=[1]
\end{lstlisting}

\subsection{Forms.Element}
\subsubsection*{serialize}
serialize()
\subsubsection*{get\_value}
get\_value()
\subsubsection*{set\_value}
set\_value(value) {
\subsubsection*{clear}
clear()
\subsubsection*{present}
present()
\subsubsection*{activate}
activate()
\subsubsection*{disable}
disable()
\subsubsection*{enable}
enable()