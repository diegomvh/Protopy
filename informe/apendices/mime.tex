\chapter{MIME}

\textbf{MIME} \label{mime}(\textit{\textbf{M}ultipurpose \textbf{I}nternet \textbf{M}ail \textbf{E}xtensions}), 
(Extensiones de Correo de Internet Multipropósito), son una serie de
convenciones o especificaciones dirigidas a que se puedan intercambiar a través
de Internet todo tipo de archivos (texto, audio, vídeo, etc.) de forma
transparente para el usuario. Una parte importante del MIME está dedicada a
mejorar las posibilidades de transferencia de texto en distintos idiomas y
alfabetos. En sentido general las extensiones de MIME van encaminadas a
soportar:
\begin{itemize}
 \item texto en conjuntos de caracteres distintos de US-ASCII
 \item adjuntos que no son de tipo texto
 \item cuerpos de mensajes con múltiples partes (multi-part)
 \item información de encabezados con conjuntos de caracteres distintos de ASCII.
\end{itemize}

Prácticamente todos los mensajes de correo electrónico escritos por personas en
Internet y una proporción considerable de estos mensajes generados
automáticamente son transmitidos en formato MIME a través de SMTP. Los mensajes
de correo electrónico en Internet están tan cercanamente asociados con el SMTP y
MIME que usualmente se les llama mensaje SMTP/MIME.[1]

En 1991 la IETF (Internet Engineering Task Force) comenzó a desarrollar esta
norma y desde 1994 todas las extensiones MIME están especificadas de forma
detallada en diversos documentos oficiales disponibles en Internet.

MIME está especificado en seis RFCs (acrónimo inglés de Request For Comments) :
RFC 2045, RFC 2046, RFC 2047, RFC 4288, RFC 4289 y RFC 2077.

Los tipos de contenido definidos por el estándar MIME tienen gran importancia
también fuera del contexto de los mensajes electrónicos. Ejemplo de esto son
algunos protocolos de red tales como HTTP de la Web. HTTP requiere que los datos
sean transmitidos en un contexto de mensajes tipo e-mail aunque los datos pueden
no ser un e-mail propiamente dicho.

En la actualidad ningún programa de correo electrónico o navegador de Internet
puede considerarse completo si no acepta MIME en sus diferentes facetas (texto y
formatos de archivo).

\section{Introducción}
El protocolo básico de transmisión de mensajes electrónicos de Internet soporta
solo caracteres ASCII de 7 bit (véase también 8BITMIME). Esto limita los
mensajes de correo electrónico, ya que incluyen solo caracteres suficientes para
escribir en un número reducido de lenguajes, principalmente Inglés. Otros
lenguajes basados en el Alfabeto latino es adicionalmente un componente
fundamental en protocolos de comunicación como HTTP, el que requiere que los
datos sean transmitidos como un e-mail aunque los datos pueden no ser un e-mail
propiamente dicho. Los clientes de correo y los servidores de correo convierten
automáticamente desde y a formato MIME cuando envían o reciben (SMTP/MIME)
e-mails.

\section*{Encabezados MIME}
\subsubsection*{MIME-Version}
La presencia de este encabezado indica que el mensaje utiliza el formato MIME. Su valor es típicamente igual a "1.0" por lo que este encabezado aparece como:

\begin{verbatim}
   MIME-Version: 1.0
\end{verbatim}

Debe señalarse que los implementadores han intentado cambiar el número de
versión en el pasado y el cambio ha tenido resultados imprevistos. En una
reunión de IETF realizada en Julio 2007 se decidió mantener el número de versión
en \emph{``1.0''} aunque se han realizado muchas actualizaciones a la versión de MIME.

\subsubsection*{Content-Type}
Este encabezado indica el tipo de medio que representa el contenido del mensaje,
consiste en un tipo: type y un subtipo: subtype, por ejemplo:
\begin{verbatim}
 Content-Type: text/plain
\end{verbatim}
A través del uso del tipo multiparte (multipart), MIME da la posibilidad de
crear mensajes que tengan partes y subpartes organizadas en una estructura
arbórea en la que los nodos hoja pueden ser cualquier tipo de contenido no
multiparte y los nodos que no son hojas pueden ser de cualquiera de las
variedades de tipos multiparte. Este mecanismo soporta:
\begin{itemize}
\item{mensajes de texto plano usando text/plain (este es el valor implícito para
el encabezado "Content-type:")}
\begin{item}
\item{texto más archivos adjuntos (multipart/mixed con una parte text/plain y
otras partes que no son de texto, por ejemplo: application/pdf para documentos
pdf, application/vnd.oasis.opendocument.text para OpenDocument text). Un mensaje
MIME que incluye un archivo adjunto generalmente indica el nombre original del
archivo con un encabezado "Content-disposition:" o por un atributo name de
Content-Type, por lo que el tipo o formato del archivo se indica usando tanto el
encabezado MIME content-type y la extensión del archivo (usualmente dependiente
del SO).}


\begin{verbatim}
Content-Type: application/vnd.oasis.opendocument.text;
    name="Carta.odt"
Content-Disposition: inline;
    filename="Carta.odt"

\end{verbatim}
\end{item}
\item{reenviar con el mensaje original adjunto (multipart/mixed con una parte
text/plain y el mensaje original como una parte message/rfc822)}
\item{contenido alternativo, un mensaje que contiene el texto tanto en texto
plano como en otro formato, usualmente HTML (multipart/alternative con el mismo
contenido en forma de text/plain y text/html)}
\item{muchas otras construcciones de mensaje}
\end{itemize}
