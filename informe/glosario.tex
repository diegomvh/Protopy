
% Ejemplos de entradas en el glosario
\glossary{name={Tesis}, description={Trabajo interminable}}
\newacronym{GNU}{GNU's Not Unix}{description={A computer operating system
composed entirely of free software.}}

\newacronym{API}{\textbf{A}pplication \textbf{P}rogramming
\textbf{I}nterface}{description={Conjunto de funciones y procedimientos (o
métodos, si se refiere a programación orientada a objetos) que ofrece cierta
biblioteca para ser utilizado por otro software como una capa de abstracción.}}

\newacronym{DOM}{\textbf{D}ocument \textbf{O}bject
\textbf{M}odel}{description={Interfaz de programación de aplicaciones que
proporciona un conjunto estándar de objetos para representar documentos HTML y
XML, un modelo estándar sobre cómo pueden combinarse dichos objetos, y una
interfaz estándar para acceder a ellos y manipularlos.}}

\newacronym{JSON}{\textbf{J}ava\textbf{S}cript \textbf{O}bject
\textbf{N}otation}{description={Formato ligero para el intercambio de datos.}}

\newacronym{RPC}{\textbf{R}emote \textbf{P}rocedure
\textbf{C}all}{description={Es un protocolo que permite a un programa de
ordenador ejecutar código en otra máquina remota sin tener que preocuparse por
las comunicaciones entre ambos.}}


\storeglosentry{linux}{name={Linux}, description={Any Unix-like computer operating system that uses the Linux kernel.}}
%\gls{linux} % displays name field of the linux entry (in this case "Linux")
%\useGlosentry{linux}{GNU/Linux} % displays "GNU/Linux"
%\GNU % displays "GNU's Not Unix (GNU)" the first time this is used
%\GNU % displays "GNU" all subsequent times
% NB: remember to use \GNU\ if want to retain the space after the acronym=======
